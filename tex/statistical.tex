In this section we derive the number of ISD calls we would need, in order to achieve some bounded error in our estimate.
Starting from this, we are able to derive a closed form formula for the time complexity of our approach.

\subsection{Statistical reliability}

We consider the statistical reliability of each measurement, that is, we define a confidence interval and estimate the probability that our estimate is within the defined interval.
To this end, we consider that the $i$-th call to ISD returns a variable $\widehat N_i$ (the number of found solutions) which is Binomial distributed, as follows
$$\mathbb{P}[\widehat N_i = x] = \binom{N_{\mathscr C}(w)}{x}p_{\mathsf{ISD}}(w)^x\big(1-p_{\mathsf{ISD}}(w)\big)^{N_{\mathscr C}(w)-x}.$$
Indeed, under Assumption \ref{ass:random_cw}, any weight-$w$ codeword behaves as a random codeword over $\mathscr S_w^*$.
Hence, it will be returned as the output of ISD with probability $p_{\mathsf{ISD}}(w)$.
So, $\widehat N_i$ follows a Binomial distribution with parameter $p_{\mathsf{ISD}}(w)$.
Since each call to ISD is independent from the others, we consider that all the $\widehat N_i$ are independent and uncorrelated variables.
To estimate the error in our estimate, we consider the following theorem, which is a straightforward application of the Chernoff's bound reported in Theorem \ref{the:chernoff}.
\textcolor{blue}{[I changed $t$ with $M$; guarda se è coerente dappertutto.]}
\begin{theorem}
Let us consider the new estimator with input parameters $w$ and $m$.
For any $\varepsilon \geq 0$, the error in the estimate is not greater than $\varepsilon$ with probability greater than $$1-e^{-m \cdot p_{\mathsf{ISD}}(w) \frac{\varepsilon^2}{3}}.$$ 
\end{theorem}
\begin{IEEEproof}
We consider that each $\widehat N_i$ is a random variable, with mean $ p_{\mathsf{ISD}}(w)  N_{\mathscr C}(w)$ and taking values in $[0 , N_{\mathscr C}(w)]$.
Remember that our estimate is obtained as
$N_{\mathscr C}^*(w) = \frac{1}{m \cdot p_{\mathsf{ISD}}(w) }\sum_{i = 1}^m\widehat N_i$.
We have that we will make an error not greater than some $\varepsilon \geq 0$ whenever 
$$(1-\varepsilon)N_{\mathscr C}(w)\leq N_{\mathscr C}^*(w)\leq(1+\varepsilon)N_{\mathscr C}(w),$$
which implies
$$m(1-\varepsilon) p_{\mathsf{ISD}}(w)  N_{\mathscr C}(w)\leq\sum_{i = 1}^m\widehat N_i\leq m(1+\varepsilon) p_{\mathsf{ISD}}(w)  N_{\mathscr C}(w).$$
%$$t(1-\varepsilon)M\leq\sum_{i = 1}^M\widehat N_i\leq t(1+\varepsilon)M.$$
Dividing all sides by $N_{\mathscr C}(w)$, we obtain
$$m(1-\varepsilon) p_{\mathsf{ISD}}(w) \leq\sum_{i = 1}^m\frac{\widehat N_i}{N_{\mathscr C}(w)}\leq m(1+\varepsilon) p_{\mathsf{ISD}}(w) .$$
Notice that the above corresponds to imposing
\begin{equation}
\label{eq:confidence1}
\left|\sum_{i = 1}^m\frac{\widehat N_i}{N_{\mathscr C}(w)} - m\cdot p_{\mathsf{ISD}}(w) \right|\leq m \varepsilon \cdot  p_{\mathsf{ISD}}(w) .
\end{equation}
In other words, our estimate will be affected by an error which is not greater than $\varepsilon$ with probability
\begin{equation}
\label{eq:pr_confidence1}
1- \mathbb P\left[\left|\sum_{i = 1}^m\frac{\widehat N_i}{N_{\mathscr C}(w)} - m\cdot  p_{\mathsf{ISD}}(w) \right|\geq m \cdot \varepsilon  p_{\mathsf{ISD}}(w) \right].
\end{equation}
We now manipulate \eqref{eq:confidence1} and \eqref{eq:pr_confidence1} so that Theorem \ref{the:chernoff} can be applied.
To this end, we first define $X_i = \frac{\widehat N_i}{N_{\mathscr C}(w)}$, which is a random variable bounded as $0\leq X_i\leq 1$ and with mean value $\mathbb E[X_i] =  p_{\mathsf{ISD}}(w) $.
Notice that $X = \sum_{i = 1}^m\frac{\widehat N_i}{N_{\mathscr C}(w)} = \sum_{i = 1}^m X_i$ and $\mu = \mathbb E[X] = \sum_{i = 1}^m\mathbb E[X_i] = m\cdot  p_{\mathsf{ISD}}(w) $.
Hence, \eqref{eq:pr_confidence1} can be rewritten as
\begin{equation}
\label{eq:pr_confidence2}
1- \mathbb P\left[\left|X - \mu\right|\geq  \mu\varepsilon\right].
\end{equation}
Applying the Chernoff's bound, we obtain
\begin{align*}
1 - \mathbb P\left[\left|X - \mu\right|\geq  \mu\varepsilon\right]&\nonumber \geq 1- e^{-\mu\frac{\varepsilon^2}{3}}\\\nonumber
& = 1-e^{-m\cdot  p_{\mathsf{ISD}}(w) \frac{\varepsilon^2}{3}}.
\end{align*}
\end{IEEEproof}

This allows us to derive a trivial result about the number of ISD calls we need, to achieve a some confidence interval with a bounded trivial.

\begin{theorem}[\textbf{Number of ISD calls for approximate counting}]\label{the:num_calls}
Let us consider the considered estimator for weight $w$ and $m$ calls to ISD.
Then, to achieve a relative error not greater than a constant $\varepsilon \geq 0$ with probability at least $1-\zeta$, with $\zeta\geq 0$ and constant, we need
$$m\geq \frac{3}{\varepsilon^2 p_{\mathsf{ISD}}(w) }\ln(1/\zeta).$$
\end{theorem}
Interestingly, we can also derive the number of ISD calls we need, in order to derive an exact estimate for $N_{\mathscr C}(w)$.
To this end, consider the following theorem.
\begin{theorem}[\textbf{Number of ISD calls for exact counting}]
Let us consider the estimator in Algorithm \ref{alg:newApproach}, for weight $w$ and $m$ calls to ISD.
Then to have that $\left\lfloor N^*_{\mathscr C}(w)\right\rceil$ is equal to $N_{\mathscr C}(w)$ with probability at least $1-\zeta$ (with $\zeta\geq 0$ and constant), we need
$$m\geq \frac{12\ln(1/\zeta) }{p_{\mathsf{ISD}}(w)}N_{\mathscr C}^2(w).$$
\end{theorem}
\begin{IEEEproof}
We observe that $\left\lfloor N_{\mathscr C}^*(w)\right\rceil = N_{\mathscr C}(w)$ if
$$N_{\mathscr C}(w) - \frac{1}{2}\leq N_{\mathscr C}^*(w)\leq N_{\mathscr C}(w) + \frac{1}{2}.$$
Since we are expressing the maximum error term, in absolute terms, as $\pm \varepsilon N_{\mathscr C}(w)$, we set 
$$\varepsilon = \frac{1}{2}  N_c(w).$$
Substituting this into the value resulting from Theorem \ref{the:num_calls}, we obtain the thesis.
\end{IEEEproof}
Putting everything together, and considering the running time of ISD algorithms, we can derive the resulting time complexity for our estimator, as a function of only i) the desired error term, ii) the confidence in the estimate, and iii) the performances of the considered ISD algorithm.
Interestingly, we can derive also a 
\begin{proposition}\textbf{Overall cost}\\
Our estimator using a ISD subroutine with cost $t_{\mathsf{ISD}}(w)$, on input $w$ and desiring a confidence interval $\varepsilon \geq 0$, has average time complexity
$$m\cdot t_{\mathsf{ISD}}(w) = \frac{3\ln(1/\zeta)}{\epsilon^2}\cdot\frac{t_{\mathsf{ISD}}(w)}{p_{\mathsf{ISD}}(w)}.$$
%Regardless of the specific ISD variant (e.g., for any algorithm operating as in Algorithm \ref{alg:isd_general}), the above is lower bounded by $$\frac{3\ln(1/\zeta)}{\epsilon^2}N_{\mathscr C}(w).$$
%If both $\zeta$ and $\epsilon$ are constant, then the lower bound is in $O\big(N_{\mathscr C}(w)\big)$.
\end{proposition}

\subsection{The cost with   Lee\&Brickell and Stern's algorithms}

Considering Lee\&Brickell ISD, we get.
\begin{proposition}\textbf{Overall cost using Lee\&Brickell's ISD}\\
Our estimator using Lee\&Brickell's ISD as subroutine takes cost
$$\frac{3\ln(1/\zeta)}{\epsilon^2}\cdot \frac{\binom{n}{w}(q-1)^p}{\binom{n-k}{w-p}}.$$
\end{proposition}
\begin{IEEEproof}
 We consider that $$t_{\mathsf{ISD}} = \frac{n(n-k-ell)^2}{p_{\mathsf{inv}(\ell)}}+t_{\mathsf{Solve}}.$$
 Remember that we need to choose $p$ so that the ratio $\frac{t_{\mathsf{ISD}}}{p_{\mathsf{ISD}}}$ is minimized.
 Since $t_{\mathsf{ISD}}$ is always at least $t_{\mathsf{PGE}} = \frac{n(n-k+\ell)^2}{p_{\mathsf{inv}}}$, and $p_{\mathsf{ISD}}$ increases with $p$ (\textcolor{blue}{[capire se vale per ogni $w$]}, the optimum will be reached with values $p\geq p^*$, where $p^*$ is such that $t_{\mathsf{Solve}}\approx t_{\mathsf{PGE}}$.
 So, we have that the numerator is $\geq 2 t_{\mathsf{Solve}} = O\left(t_{\mathsf{Solve}}\right)$.
 Considering the expressions for $t_{\mathsf{Solve}}$ and $p_{\mathsf{ISD}}$ and doing some simple algebraic manipulations, we obtain the formula in the thesis.
\end{IEEEproof}
We can derive analogous results for Stern's ISD.
\begin{proposition}\textbf{Overall cost using Stern's ISD}\\
Our estimator using Stern's ISD as subroutine takes cost
$$t\cdot t_{\mathsf{ISD}}(w) = \frac{3\ln(1/\zeta)}{\epsilon^2}\cdot\frac{t_{\mathsf{ISD}}(w)}{p_{\mathsf{ISD}}(w)}.$$
\end{proposition}
\begin{IEEEproof}
We first rewrite the success probability as
$$p_{\mathsf{ISD}}(w) = \frac{\binom{n}{w}(q-1)^{2p}}{\binom{n-k-\ell}{w-2p}L^2}.$$
Consider that $t_{\mathsf{Solve}} = 2L+L^2q^{-\ell}$, so that
\begin{align*}
p_{\mathsf{ISD}}(w)t_{\mathsf{Solve}} &\nonumber = \frac{\binom{n}{w}(q-1)^{2p}}{\binom{n-k-\ell}{w-2p}L}\big(2+Lq^{-\ell}\big).
\end{align*}
\textcolor{blue}{[TO BE COMPLETED]}
\end{IEEEproof}

To consider some examples, we consider the complexities on some example parameters.
\textcolor{blue}{[Include figures with comparison between Stern and Lee Brickell, for random codes.]}
