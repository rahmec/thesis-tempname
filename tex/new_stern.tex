We describe how we can improve the cost of the estimator.
Basically, we propose to modify the Stern $\mathsf{Solve}$ subroutine.
Namely, instead of using the full Hamming sphere of vectors with weight $p$ and length $\frac{k+\ell}{2}$, we propose to use only a randomly selected subset of the sphere.
In other words, we assume that each list in Stern's algorithm has size
$$L = \chi |S_{\frac{k+\ell}{2}, p}|,$$
where $\chi$ is some constant that we will determine in the following.
To have amortized lists, we will choose $q^\ell = L$, leading to
$$\ell = \log_q(\chi)+\log_q(|S_{\frac{k+\ell}{2},p}|) = \log_q(\chi)+\frac{k+\ell}{2}h_q\left(\frac{2p}{k+\ell}\right),$$
where $h_q:\mathbb [0 ; 1]\rightarrow [0 ; 1]$ is the $q$-ary entropy function.
Let $\ell = \psi n$, where $\psi \in [0 ; 1-R]$, and $\rho = p/n$.
We notice that 
$h_q\left(\frac{2p}{k+\ell}\right) = h_q\left(\frac{}{}\right)$
So far we have provided a very high level description of our idea. In the following we adapt it in the context of Stern algorithm.

\subsection{Using Stern for counting codewords}
In this section we describe a new approach for counting the number of codewords of a given weight $w$. Since we are working with a collision search approach, we need to take into account three main factors:
\begin{itemize}
    \item \textit{initial permutation}: we must consider that we are able to represent only the solutions that allow a splitting as in (\emph{\textcolor{red}{inserire una breve discussione sul collision search}}).
    This can be estimated by first obtaining the probability that the initial permutation is valid for a specific codeword. We can compute this probability simply by counting, given a $w$-codeword, how many ways can one break it into blocks of length $n'/2$ such that each block has weight $p/2$. This leads to the following probability:
    \begin{equation}
    P_\pi(p) = \frac{\binom{n/2}{p/2}^{2}\binom{n-n'}{w-p}}{\binom{n}{w}}.
    \end{equation}
    Since we assume to start from $N_{\mathcal{C}(w)}$ solutions for the initial instance, the number of codewords that get mapped into solutions for the small instance is $$N' = N_{\mathcal{C}(w)} P_\pi(a,p);$$
    \item \textit{starting with lists of size $<L$}: remember that, as a degree of freedom, we can have $X<L$. 
    So, we have to consider also that each initial list contains the corresponding split of $\6c'$.
    That is, we want to consider the probability that $\6x^{(0)}_j\in\mathcal E_j^{(0)}$, for all $j$.
    It is easy to see that, for any $j$, it holds that
    \begin{equation}
    \mathrm{Pr}\left[\6x^{(0)}_j\in \mathcal E_j^{(0)}\right] = \frac{X}{L} = \frac{X}{\binom{n/2}{p/2}(q-1)^{p/2}}.
    \end{equation}
    So, the probability that a solution is indeed enumerated, in the initial lists, is
    \begin{align}
    P_{\mathcal E}(p) = \frac{X^{2}}{\binom{n/2}{p/2}^{2}(q-1)^p}.
    \end{align}
    \item \textit{partial sums}: in every level, we need to guess some partial sums.
    All the solutions such that the considered splitting does not correspond to the desired partial sums will be filtered out.
    Let us consider that, for each of the $X$ solutions, to produce level $i$ we need to guess correctly $2^{a-i}-1$ partial sums.
    So, the probability that the considered partial sums are valid, for a solution, is
    \begin{align*}
    P_{\sf{Guess}}(m,a) = \prod_{i = 1}^{a-1}q^{-m\big(2^{a-i}-1\big)} = q^{-m(2^a-a-1)}.
    \end{align*}
    Because of amortized lists, we have $q^m = X$, so that
    $$P_{\sf{Guess}}(m,a) = X^{-(2^a-a-1)} = 2^{-\chi(2^a-a-1)n}.$$
\end{itemize}
On average, the number of solutions we produce is
\begin{align}
M &\nonumber = N_{\mathcal{C}}(w) \cdot P_\pi(a,p)\cdot P_{\mathcal E}(a,p,\chi)\cdot P_{\sf{Guess}}(m,a)\\\nonumber
& = N_{\mathcal{C}}(w) \frac{X^{2}\binom{n-n'}{w-p}}{\binom{n}{w}(q-1)^p}.
\end{align}
Since we can estimate $M$ simply by executing the algorithm, we can consequently find an estimate for the number of codewords of weight $w$, obtaining:
\begin{equation}
    N_{\mathcal{C}}(w) = M \cdot \frac{\binom{n}{w}(q-1)^p}{X^{2}\binom{n-n'}{w-p}}.
\end{equation}
