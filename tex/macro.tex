\usepackage{xcolor}

\newcommand{\ep}[1] {\textcolor{blue}{Edo: #1}}
\newcommand{\rev}[1] {\textcolor{red}{#1}}

\usepackage{amsfonts,amsmath,amssymb,amsthm}
\usepackage{setspace}
\usepackage[noend]{algpseudocode}
%\usepackage{algorithm}
%\usepackage{algorithmicx}
\usepackage{xspace}
\usepackage{comment}
\usepackage{graphicx}
\usepackage{multirow}
\usepackage{algorithmicx}


\newcommand{\F}{\mathbb{F}_2}
\newcommand{\Fq}{\mathbb{F}_q}
\newcommand{\C}{\mathcal{C}}
\newcommand{\jc}[1]{{\color{red}\textbf{JCD}: \textit{#1}}}
\newcommand{\code}{\mathscr C}

\newcommand{\dgv}{\ensuremath{d_{GV}}}

\newcommand{\wt}{\mathrm{wt}}
\newcommand{\etal}{\emph{et al.}\xspace}
\newcommand{\ie}{\emph{i.e.}\xspace}
\newcommand{\eg}{\emph{e.g.}\xspace}
\newcommand{\lar}{\stackrel{\mathdollar}{\leftarrow}}
\newcommand{\sk}{\ensuremath{\mathsf{sk}}\xspace}
\newcommand{\pk}{\ensuremath{\mathsf{pk}}\xspace}
\newcommand{\accept}{\ensuremath{\mathsf{Accept}}\xspace}
\newcommand{\reject}{\ensuremath{\mathsf{Reject}}\xspace}
\newcommand{\WRH}{\textsf{WRH}}
\newcommand{\6}{\mathbf}
\newcommand{\prob}[1]{\mathrm{Pr}\left[#1\right]}
\newcommand{\supp}{\mathrm{Supp}}


\algrenewcommand\algorithmicrequire{\textbf{Input:}}
\algrenewcommand\algorithmicensure{\textbf{Output:}}

\usepackage{url}

\usepackage{mathrsfs}

\newtheorem{definition}{Definition}
\newtheorem{remark}{Remark}
\newtheorem{theorem}{Theorem}
\newtheorem{assumption}{Assumption}
\newtheorem{proposition}{Proposition}

\usepackage{calligra}
%\newcommand{\distrib}{{\small{\text{\calligra{S} }}}}
\newcommand{\distrib}{\mathcal{U} }
\newcommand{\unif}{{\small{\text{\calligra{U} }}}}
\newcommand{\bernoulli}{{\small{\text{\calligra{B} }}}}
\newcommand{\ldpc}{{\small{\text{\calligra{L} }}}}
\newcommand{\1}[1]{{\mathcalboondox{#1} }}
\newcommand{\isd}{\mathcal{ISD}}



\DeclareFontFamily{U}{BOONDOX-calo}{\skewchar\font=45 }
\DeclareFontShape{U}{BOONDOX-calo}{m}{n}{
  <-> s*[1.05] BOONDOX-r-calo}{}
\DeclareFontShape{U}{BOONDOX-calo}{b}{n}{
  <-> s*[1.05] BOONDOX-b-calo}{}
\DeclareMathAlphabet{\mathcalboondox}{U}{BOONDOX-calo}{m}{n}
\SetMathAlphabet{\mathcalboondox}{bold}{U}{BOONDOX-calo}{b}{n}
\DeclareMathAlphabet{\mathbcalboondox}{U}{BOONDOX-calo}{b}{n}

\usepackage{scalerel}


\usepackage[linesnumbered, ruled, noend]{algorithm2e}

\newcommand\mycommfont[1]{\footnotesize\ttfamily\textcolor{blue}{#1}}
\SetCommentSty{mycommfont}

\usepackage{tikz}

\tikzset{
  symbol/.style={
    draw=none,
    every to/.append style={
      edge node={node [sloped, allow upside down, auto=false]{$#1$}}}
  }
}

\usepackage{tikzit, tikz-cd, tkz-graph}
\usepackage{tikz}

\tikzset{
  symbol/.style={
    draw=none,
    every to/.append style={
      edge node={node [sloped, allow upside down, auto=false]{$#1$}}}
  }
}

\newcommand{\functionFive}[5]{%
  \begin{tikzcd}[
    column sep=2em,
    row sep=0ex,
    ampersand replacement=\&
  ]
  #1\colon \&[-3em]
  #2\vphantom{#3} \arrow[r] \&
  #3\vphantom{#2} \\
  \&
  #4\vphantom{#5} \arrow[r,mapsto] \&
  #5\vphantom{#4}
  \end{tikzcd}%
}

\newcommand{\functionTwo}[7]{%
  \begin{tikzcd}[
    column sep=2em,
    row sep=0ex,
    ampersand replacement=\&
  ]
  #1\colon \&[-3em]
  #2\vphantom{#3} \arrow[r] \&
  #3\vphantom{#4} \arrow[r] \&
  #4\vphantom{#3} \\
  \&
  #5\vphantom{#6} \arrow[r,mapsto] \&
  #6\vphantom{#7} \arrow[r,mapsto] \&
  #7\vphantom{#6}
  \end{tikzcd}%
}


\usepackage{amsmath}
\usepackage{mleftright}