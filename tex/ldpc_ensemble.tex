We start by considering a family of LDPC codes which we use as a prototype to introduce our analysis.
We start with the following definition.
\begin{definition}\label{def:family_one}
For $r,n\in\mathbb N$, $1\leq r\leq n$ and $\rho\in[\![ 0 ; 1]\!]$, we define $\1{D}_{r,n}(\rho)$ as the distribution over $\mathbb F_2^{r\times n}$ such that, if $\6H\sim \1{D}_{r,n}(\rho)$, then 
$$h_{i,j}\sim \1{B}(\rho),\hspace{2mm}\forall i\in[1;r],\hspace{1mm}j\in[1;n].$$
\end{definition}
In other words, to sample a matrix from $\1{D}_{r,n}(\rho)$, one simply has to generate each of its elements according to $\1{D}_{r,n}(\rho)$.
When $\rho \ll \frac{1}{2}$, then a typical matrix sampled from the distribution can be seen as the parity-check matrix of an LDPC code.
For instance, a matrix sampled from $\1{D}_{r,n}(\rho)$ is such that its expected Hamming column and row weights are, respectively, $v_r = \rho r$ and $v_c = \rho n = \frac{n}{r}v_c$.
We then see that there is a clear analogy between a matrix $\6H\sim\1{D}_{r,n}(\rho)$ and the parity-check matrix of a regular LDPC codes.
Note that any matrix in $\mathbb F_2^{r\times n}$ can be sampled from $\1{D}$, but a typical matrix coming from $\1{D}$ is expected to be, basically, a sparse matrix.

To turn the distribution $\1{D}_{r,n}(\rho)$ into a distribution of parity-check matrices for codes with redundancy $r$, we need a technical refinement, which we introduce next.
\begin{definition}\label{def:family_one}
For $r,n\in\mathbb N$, $1\leq r\leq n$ and $\rho\in[\![ 0 ; 1]\!]$, we define $\widetilde{\1{D}}_{r,n}(\rho)$ as the distribution over $\mathbb F_2^{r\times n}$ which arises from the following experiment:
\begin{itemize}
    \item[1. ] sample $\6H$ according to $\1{D}_{r,n}(\rho)$;
    \item[2. ] output $\6H$ if $\mathrm{rank}(\6H) = r$, otherwise restart from step 1.
\end{itemize}
\end{definition}
It is easily seen that any matrix sampled from $\widetilde{\1{D}}_{r,n}(\rho)$ defines the parity-check matrix of a code with redundancy $r$ and length $\leq n$.
In the next section we derive the average weight distribution of the codes generated from $\widetilde{\1{D}}_{r,n}(\rho)$.

\subsection{Average weight distribution over $\widetilde{\1{D}}_{r,n}(\rho)$}

We start with the following theorem, where we derive the probability that a random weight-$d$ codeword is in the kernel of a matrix sampled from $\1{D}_{r,n}(\rho)$.
\begin{theorem}\label{the:pr_c_H}
Let $\6c\in S_{n,d}$ and $\6H\sim\1{D}_{r,n}(\rho)$.
Then
$$u_{r,n}(d,\rho) = \mathrm{Pr}[\6c\in C(\6H)] = \left(\sum_{\begin{smallmatrix}
i \in [0 ; d]\\
\text{$i$ even}\end{smallmatrix}}\binom{d}{i}\rho^i(1-\rho)^{d-i}\right)^r.$$
\end{theorem}
\begin{IEEEproof}
We are going to have $\6c\in C(\6H)$ if and only if, for every row $\6h$ of $\6H$, we have $\6h \6c^\top = 0$.
Since $\6H$ is sampled at random from $\1{D}_{r,n}(\rho)$, every row is constituted by $n$ independent and uncorrelated random variables with distribution $\1{D}(\rho)$.
Then, we have
$$\mathrm{Pr}[ \6h \6c^\top = 0] = \sum_{\begin{smallmatrix}
i \in [0 ; d]\\
\text{$i$ even}\end{smallmatrix}}\binom{d}{i}\rho^i(1-\rho)^{d-i}.$$
Since there are $r$ rows, we consider the $r$-th power of the above quantity to obtain the desired probability.
\end{IEEEproof}
\begin{theorem}\label{the:average_weight_H}
Let $\6H\sim\1{D}_{r,n}(\rho)$; then
$$\langle |C_d(\6H)|\rangle = g_{r,n,\rho}(d) =  \binom{n}{d}u_{r,n}(d,\rho)$$
\end{theorem}
\begin{IEEEproof}
Let $f(\6c,\6H)$ be the function that returns $1$ if $\6c\in C(\6H)$, and $0$ otherwise.
We then have
$$\langle |C_d(\6H)|\rangle = \sum_{\6H\in \mathbb F_2^{r\times n}}\mathrm{Pr}[\6H\mid \6H\sim\1{D}_{r,n}(\rho)]\sum_{\6c\in S_{n,d}} f(\6c,\6H),$$
where $\mathrm{Pr}[\6H\mid \6H\sim\1{D}_{r,n}(\rho)]$ is the probability that the distribution $\1{D}_{r,n}(\rho)$ outputs $\6H$. 
With a simple rewriting, we have
\begin{align*}
\langle |C_d(\6H)|\rangle &\nonumber = \sum_{\6c\in S_{n,d}}\sum_{\6H\in \mathbb F_2^{r\times n}}\mathrm{Pr}[\6H\mid \6H\sim\1{D}_{r,n}(\rho)] f(\6c,\6H)]\\\nonumber
& = \sum_{\6c\in S_{n,d}}\sum_{\begin{smallmatrix}\6H\in\mathbb F_2^{r\times n}\\
\6c\in C(\6H)\end{smallmatrix}}\mathrm{Pr}[\6H\mid \6H\sim\1{D}_{r,n}(\rho)]\\\nonumber
& = \sum_{\6c\in S_{n,d}}u_{r,n}(d,\rho).
\end{align*}
Since $u_{r,n}(d,\rho)$ is independent of $\6c$, it is enough to multiply such a quantity by the number of elements in $S_{n,d}$, which is given by $\binom{n}{d}$.
\end{IEEEproof}
Notice that the quantity expressed in Theorem \ref{the:average_weight_H} is not exactly the average weight distribution of codes with redundancy $r$.
Indeed, remember that $\1{D}_{r,n}(\rho)$ is a distribution of matrices with $r$ rows, but there is no guarantee about the rank.
To consider the rank, we need to make use of the distribution $\tilde{\1{D}}_{r,n}(\rho)$.
However, unless extreme parameters are considered, it is very likely that a matrix sampled from $\1{D}_{r,n}(\rho)$ has full rank $r$, so that we can safely employ $g_{r,n,\rho}(d)$ for the average weight distribution of matrices sampled from $\tilde{\1{D}}_{r,n}(\rho)$.
We formalize such an assumption in the following.
\begin{assumption}\label{ass:ass_rate}
We assume that the distributions $\1{D}_{r,n}(\rho)$ and $\tilde{\1{D}}_{r,n}(\rho)$ are indistinguishable. 
Then, if $\6H\sim\tilde{\1{D}}_{r,n}(\rho)$, we can use $g_{r,n,\rho}(d)$ to obtain $\langle |C_d(\6H)|\rangle$.
\end{assumption}
\begin{remark}
\emph{The validity of the above assumption will be confirmed in the remainder of the paper, where we make use of numerical simulations to confirm our treatment.
Yet, we can also give some theoretical arguments to sustain it.
For the sake of simplicity, let us consider the case of $\rho = 1/2$.
The probability that $\6H\sim\1{D}_{r,n}(\rho)$ does not have full rank is given by
$\prod_{i = 1}^{r-1}(2^n-2^i)$.
If $n = r$, then this probability (for sufficiently large $n$) tends to $0.288$.
So, one can distinguish between $\1{D}_{n,n}(\rho)$ and $\tilde{\1{D}}_{n,n}(\rho)$ rather easily: it is enough to  sample a large enough number of matrices and check whether they have full rank or not.
However, for $n>r$, then the probability to obtain a non full rank matrix from $\1{D}_{n,n}(\rho)$ becomes negligible.
Indeed, it is easily seen that for large $n$ and $r$ sufficiently lower than $n$, the previous probability asymptotically tends to $1$.}
\end{remark}